\documentclass{article}

\usepackage[utf8]{inputenc}
\usepackage{graphicx}
\usepackage[T1]{fontenc}

% Path relative to the main .tex file.
\graphicspath{ {./images/} }

\title{Scalavelli project}

\date{14 Agosto - 15 Settembre 2020}

\author{
Cavalluzzo M.
\and
Giorgetti L.
\and
Pagnini L.
\and
Tentoni D.
}

\begin{document}
    \pagenumbering{gobble} % no numbers.
    \maketitle
    \newpage
    \pagenumbering{arabic} % start counting numbers.

    \begin{abstract}
        Progetto scritto in Scala per giocare al celebre gioco Machiavelli sul proprio computer e sfidare altri giocatori.
    \end{abstract}

    \tableofcontents

    \newpage


    \section{Processo di sviluppo}\label{sec:processo-di-sviluppo}

    \subsection{Metodologia}
    Sin da subito abbiamo deciso di adottare una metodologia di sviluppo \textit{Agile-Scrum}, seppur non scegliendo un vero e proprio Scrum Master.
    Chi più e chi meno, a seconda dei vari sprint, ognuno ha avuto l'occasione e il modo di ricoprire tale ruolo.
    In questo modo tutti hanno potuto dare il loro contributo per la riuscita del progetto, coordinando il team con l'aiuto degli strumenti a disposizione.
    In accordo con tale metodologia, il lavoro è stato suddiviso in \textit{Sprint}, della durata media di una settimana e mezzo.
    Allo scadere del tempo si sarebbe dovuti arrivare a sviluppare un numero minimo di funzionalità dell'applicativo.
    I meeting sono stati frequenti all'inizio e ne sono stati svolti alcuni saltuariamente all'interno di ogni sprint.
    Questo perché, alternandosi con un periodo di lavoro autonomo, abbiamo ritenuto necessario e producente confrontarsi anche durante gli sprint su scelte sintattiche e pattern di sviluppo tra tutti i membri del team, in modo da condividere conoscenze ed entusiasmo.

    \subsection{Strumenti adottati}

    \paragraph{VCS}
    Si è deciso di utilizzare \textit{Git} per effettuare il versioning del codice durante lo sviluppo attraverso la piattaforma \textit{GitHub}.
    L’utilizzo che ne abbiamo fatto è descritto nell'immagine di seguito:
    \begin{center}
        \includegraphics[scale=0.5]{git-workflow-1-1}
    \end{center}
    Il branch di default è sempre il \textit{master}, al quale \textit{dev} è sempre allineato.
    Nel caso in cui debbano essere prodotti degli hotfix, essi vengono svolti su un branch che parte dal \textit{master} e vi ritorna subito, senza passare da \textit{dev}.
    Ogni volta che si implementava una nuova funzionalità o si risolveva una fix, veniva creato una branch a parte.
    Successivamente, a lavoro ultimato, veniva una creata pull request per mergiare sul branch \textit{dev}.
    Ogni pull request su \textit{dev} deve essere approvata almeno da un altro membro del gruppo.
    Per mergiare invece un insieme di funzionalità sviluppate e testate dal branch \textit{dev} al \textit{master}, deve sempre essere aperta una pull request, che deve essere approvata da tutti i membri del gruppo.
    Questo per essere certi che tutti i membri del gruppo siano consapevoli del lavoro svolto da tutti gli altri.

    \paragraph{Build Automation}
    Si è deciso di usare \textit{SBT}, dato che durante il corso lo si é sempre preferito rispetto al 'cugino' \textit{Gradle} per lo svolgimento degli elaborati, nonostante il fatto che alcuni componenti del gruppo lo utilizzino in ambito aziendale.

    \paragraph{Continuous Integration}
    Si è deciso di usare \textit{Travis CI}.
    Consisteva nell'unica soluzione freeware che i componenti del gruppo abbiano mai usato, introdotta proprio in questo corso.
    Tuttavia, è stato necessario approfondire l'argomento tramite studio autonomo, dato che ciò che si era appreso a lezione é stato ritenuto insufficiente per la buona riuscita del progetto per come è stato pensato.
    Ad esso é stata adattata l'esecuzione dei test necessari per verificare la correttezza del lavoro svolto e poter poi effettuare dei rilasci senza regressioni.
    Tramite lo stesso servizio é stata effettuata la \textit{Continuous Delivery}, rilasciando dei pacchetti compilati su \textit{Github Releases}.

    \newpage


    \section{Requisiti}\label{sec:requisiti}
    \subsection{Business}\label{subsec:business}
Il progetto Scalavelli vuole ricreare l’esperienza del celebre gioco di carte Machiavelli (Ramino Machiavellico) in modalità multiplayer, tra giocatori differenti collocati sulla stessa macchina o nella stessa LAN. Ogni giocatore deve potersi identificare con un proprio username e connettersi ad una lobby.
Essa deve poter contenere da 2 a 6 giocatori.
Raggiunto il numero necessario di essi si potrà partecipare ad una partita.
Vi è anche la possibilità di scegliere di giocare specificatamente con i propri amici inserendo il codice identificativo per una partita privata.

\subsection{Utente}\label{subsec:utente}
L’utente medio utilizzatore potrà interagire solamente con il client dell’applicazione.
Le operazioni in fase di creazione della partita sono:
\begin{itemize}
    \item Specificare il proprio username (per potersi registrare nel server) e il numero di giocatori di una partita pubblica;
    \item Specificare il proprio username e il codice di una partita privata che gli è stato inviato;
    \item Specificare il proprio username, il numero di giocatori e generare un codice per una partita privata;
    \item Inserire o creare una lobby privata;
    \item Connettersi ad una lobby e attendere il raggiungimento del numero di giocatori.
\end{itemize}
Dopo aver generato una partita ed esservi entrato, in una nuova schermata il giocatore potrà:
\begin{itemize}
    \item Vedere sullo schermo le combinazioni che ci sono attualmente sul tavolo da gioco, le carte che ha in mano, il nome degli altri giocatori e il numero di carte nelle loro mani;
    \item Solamente nel proprio turno, eseguire le seguenti azioni:
    \begin{itemize}
        \item Giocare una combinazione: si possono mettere sul tavolo da giocare una sequenza di carte che compongono una combinazione tra un tris, un poker oppure una scala.
        Se la combinazione è valida, allora vengono tolte le carte che si vuole giocare dalla mano e vengono messe sul tavolo;
        \item Aggiungere carte ad una combinazione sul tavolo già esistente: si possono scegliere delle carte dalla propria mano, che non necessariamente compongono una combinazione valida e mettere assieme alle carte che compongono un’altra combinazione, a patto che venga sempre rispettata la validità della combinazione;
        \item Prendere delle carte dal tavolo: si possono scegliere delle carte che appartengono ad una combinazione presente sul tavolo da gioco e metterle nella propria mano.
        \item Passare il turno al giocatore successivo: dopo che viene eseguita questa azione, il turno viene passato al giocatore successivo nell’ordine.
        Il giocatore che ha eseguito questa azione non ne può effettuare nessun’altra fino a che non riprende il proprio turno dall’azione del giocatore precedente nell’ordine.
        Nel caso in cui il giocatore di turno:
        \begin{itemize}
            \item non avesse eseguito nessuna mossa;
            \item avesse in mano delle carte che ha preso dal tavolo da gioco senza averle rigiocate;
            \item se non si trovasse con meno carte in mano rispetto a quando ha iniziato il turno;
            \item se scade il tempo a disposizione;
        \end{itemize}
        allora è costretto a pescare una carta dal mazzo principale.
    \end{itemize}
\end{itemize}

\subsection{Funzionali}
Connessione al server e creazione della lobby.
Il sistema dovrà:
\begin{itemize}
    \item Permettere al client di connettersi a internet;
    \item Una volta connessi al server permettersi di unirsi ad una lobby attraverso il proprio username;
    \item Assicurarsi che il server mantenga sempre attiva la connessione con il client, per poter inviare e ricevere messaggi inerenti alla partita;
    \item Assicurarsi che qualora venga raggiunto il numero di giocatori necessario, il server dovrà subito far partire la partita tra i giocatori;
    \item Nel caso in cui un giocatore venga disconnesso dalla partita, sia una disconnessione volontaria o un problema dell’infrastruttura di rete, terminare la partita per tutti, venendo visto come un abbandono;
    \item Al termine di una partita permetterne di iniziarne una nuova.
\end{itemize}
Il gioco.
\begin{itemize}
    \item Preparazione del gioco.
    All’inizio della partita vengono svolte le seguenti azioni per permettere a tutti i giocatori di partecipare:
    \begin{itemize}
        \item Vengono mischiati due mazzi da 52 carte ognuno;
        \item Vengono distribuite 13 carte ad ogni giocatore.
    \end{itemize}
    \item Fornire ad ogni giocatore nel proprio turno un tempo massimo di 2 minuti.
    Al termine di questo, tutte le mosse effettuate vengono annullate e il giocatore pesca una carta passando il turno al giocatore successivo.
    \item Interfaccia utente: ogni giocatore deve vedere:
    \begin{itemize}
        \item le proprie carte in mano;
        \item il tavolo da gioco con le varie combinazioni;
        \item gli altri giocatori, con i loro nomi e il numero di carte nelle loro mani.
    \end{itemize}
\end{itemize}

\subsection{Non Funzionali}

\subsubsection{Scalabilità}
Il server deve essere in grado di supportare un numero indefinito di utenti nell’insieme di tutte le partite.
Nel caso in cui non ci siano abbastanza giocatori in coda per iniziare una partita, allora devono essere lasciati in attesa dell’arrivo di altri giocatori che vogliano giocare anche loro.
Il sistema non consente l’accesso ad altri utenti per una partita specifica quando questa ha raggiunto il numero di partecipanti richiesto.

\subsubsection{Modularità}
Il progetto è stato pensato in modo tale da dover effettuare meno modifiche possibili al client nel caso in cui debba essere cambiato il server e viceversa.
Nello specifico: se il server deve subire un aggiornamento, esso dovrebbe essere spento e poi riacceso.
Gli utenti finali non devono eseguire nessuna operazione sui loro client, o almeno l’aggiornamento deve essere mantenuto il più piccolo possibile.
Viceversa, nel caso di aggiornamento dei client, non dovrebbe essere necessario né riavviare il server, né aggiornare nessuna sua parte.
Tutto il software deve riuscire a funzionare anche cambiando l’implementazione interna del modulo del “core” (regole e logica di gioco) a patto di non impattare sull’interfaccia dello stesso.

\subsubsection{Usabilità}
Il sistema deve fornire agli utenti un'interfaccia chiara, semplice, ben organizzata in modo da poter utilizzare al meglio tutte le sue funzionalità messe a disposizione e visualizzate.

\subsubsection{Reattività}
Il sistema deve poter consentire di giocare una partita senza ritardi e/o blocchi temporali dati dall’esecuzione degli algoritmi di validazione e controllo e dai protocolli di comunicazione.

\subsubsection{Sicurezza}
Il sistema deve aver sempre la possibilità di controllare lo stato attuale del gioco in modo da impedire che alcuni giocatori possano iniettarne uno non veritiero nella partita attuale.
Per questo ad ogni fine turno deve essere validato da parte del server in modo che un client malevolo non possa rovinare l’esperienza di gioco agli altri giocatori.

\subsection{Implementativi}
Di seguito vengono descritti i vincoli che abbiamo cercato di rispettare per l’intero sviluppo del progetto:
\begin{itemize}
    \item Mantenere un approccio il più funzionale possibile;
    \item Mantenere, in tutti i casi in cui era possibile, una stato immutabile nei vari componenti del sistema.
\end{itemize}
Principali tecnologie e modelli utilizzati sono:
\begin{itemize}
    \item Scala: Il sistema deve essere prevalentemente sviluppato in scala.
    \item ScalaFX: Il sistema deve disporre di un’interfaccia grafica per poter interagire con il gioco e il server.
    \item Prolog: Utilizzo del prolog per implementare l’ordinamento delle carte (per seme, valore e colore), le entità del gioco e la validazione della correttezza delle combinazioni di carte.
    \item Akka: Utilizzo di varie componenti per implementare un server che risponda alle chiamate svolte dai client.
    \item TDD (Test Driven Development): Ci siamo ispirati a questa tecnica di sviluppo per scrivere un codice più efficiente, efficace e il più possibile esente da problemi.
\end{itemize}
\newpage


    \section{Design Architetturale}\label{sec:design-architetturale}

    \subsection{Model}

    \subsection{View}

    \subsection{Controller}

    \newpage


    \section{Design di Dettaglio}\label{sec:design-di-dettaglio}
    \subsection{Organizzazione del codice}
La suddivisione del codice rispecchia fortemente il design architetturale generale descritto in precedenza.
\begin{center}
    \includegraphics[scale=0.5]{moduli.png}
\end{center}
Sono stati realizzati 4 moduli:
\begin{itemize}
    \item Core: comprende tutte le entità necessarie alla realizzazione del gioco e le regole per poter portare avanti una partita, è l’unico tra i moduli totalmente indipendente dagli altri.
    \item Common: codice comune alle parti di client e server, per la maggior parte definisce i messaggi utilizzati per lo scambio di informazioni
    \item Client: contenente tutto ciò che concerne il client, interfaccia utente, parte di comunicazione con il server e di aggiornamento dello stato della partita
    \item Server: contenente tutta la parte di gestione delle lobby e delle partite in corso
\end{itemize}

\subsection{Core}
Il core modella al suo interno tutte le entità del gioco Machiavelli reale, come le carte, il mazzo e il tavolo da gioco, come anche la GameInterface, cioè un insieme di funzioni che gli altri moduli del progetto devono usare per poter interagire con le entità.
Tali funzioni infatti modellano tutte le azioni che un giocatore può svolgere nel gioco reale.
\textparagraph E’ importante sottolineare il fatto che sia immutabile e quindi non conserva alcuno stato interno.
Esso deve essere utilizzato all’esterno sfruttando appunto le API messe a disposizione.

\subsubsection{Entità}
\begin{center}
    \includegraphics[width=\textwidth]{classi-Page-1.png}
\end{center}
Le entità di gioco sono elencabili come:
\begin{itemize}
    \item Card, corredata da un Rank (valore nominale), un Seme e un Colore.
    E’ stato modellato anche il caso del Rank Asso come 14-esimo possibile valore, per poterne effettuare la validazione qualora si trovasse dopo il Re (13-esimo valore) in una combinazione
    \item Player, composta da un username, un id e una mano
    \item Hand, che contiene i metodi per poter aggiungere (o rimuovere) carte dal tavolo alla mano (dalla mano al tavolo) e per ordinare le carte
    \item CardCombination, rappresenta un tris, poker o una scala ordinata
    \item Deck, rappresenta il mazzo
    \item Board, rappresenta il tavolo di gioco e contiene un insieme di CardCombiantion e i relativi metodi per poterle aggiungere, rimuovere e aggiornare
\end{itemize}

\subsubsection{Prolog}
Per la gestione delle regole di validazione, sii è deciso di utilizzare questo linguaggio in quanto è possibile esprimerle in maniera totalmente dichiarativa conciso ed efficente.
La libreria utilizzata è TuProlog.
\begin{center}
    \includegraphics[scale=0.5]{classi-Page-2.png} %TODO: Aggiungere grafico.
\end{center}
Di seguito vengono descritte le classi:
\begin{itemize}
    \item il PrologGame espone tutte le funzionalità implementate attraverso tale linguaggio.
    Ogni qualvolta che si deve eseguire una funzionalità in Prolog, è necessario richiamare un metodo di questa classe corrispondente all’azione di Prolog.
    In particolare permette di creare le carte corredate da un valore, un seme e un colore per formare il deck di gioco, di eseguire la validazione di una sequenza di carte, che sia essa una scala, tris o poker e ne esegue l’ordinamento per seme e per valore.

    \item il PrologEngine esegue effettivamente le azioni di Prolog.
    Si è deciso di realizzare un piccolo DSL che permettesse di facilitare l’utilizzo della libreria TuProlog e di aumentarne l’espressività del codice.
    Dopo aver caricato la specifica teoria, il PrologEngine esegue le funzioni in grado di:
    \begin{itemize}
        \item risolvere un singolo obiettivo o più obiettivi
        \item verificare se un obiettivo ha successo
        \item se vi sono altre soluzioni dopo averne trovata almeno una
        \item estrarre i valori dalle variabili, dopo l’esecuzione di un predicato, tramite il metodo bindingVars
    \end{itemize}

    \item il PrologGameConverter espone le funzioni in grado di formulare obiettivi nel giusto ‘formato’ in Prolog e di convertire il risultato ottenuto dal PrologEngine nel tipo corretto, a seconda dell’utilizzo.
    In particolare, grazie all’utilizzo dell’oggetto PrologUtils, espone metodi in grado di ‘pulire’ (da caratteri non conformi) il risultato del Prolog dopo averlo convertito in stringa.
    Questo è stato reso necessario poiché quando si dava in input un obiettivo che conteneva una lista di tuple (ogni carta è una tupla che contiene nel seguente ordine valore, seme e colore), il risultato risultava essere ‘sporco’ da caratteri estranei rispetto al predicato dato in input.
    Di fatto l’oggetto PrologUtils espone delle funzioni e una Regex che lavorano su stringhe e sostituiscono i caratteri non accettati.
    La classe converter permette infine di gestire la validazione di specifici casi, ad esempio una combinazione contenente uno o più assi.
    Questo perché l’asso in una combinazione che forma una scala, può essere posto prima del due o dopo del re e quindi a fini di validazione, per la teoria definita, l’asso può assumere sia il valore 1, sia il 14.
    Il metodo OptionalValueAce gestisce i casi appena descritti cambiandone il valore in modo conforme.

\end{itemize}

\subsubsection{Game Interface}
\begin{center}
    \includegraphics[scale=0.5]{classi-Page-2.png}
\end{center}

\subsection{Client}

\subsubsection{MVC}

\subsubsection{Lobby}

\subsubsection{Game}

\subsection{Server}

\subsubsection{Lobby}

\subsubsection{Game}

\newpage


    \section{Implementazione}\label{sec:implementazione}

    \subsection{Matteo}

    \subsection{Luca}

    \subsection{Lorenzo}

    \subsection{Daniele}

    \subsubsection{Continuous Integration}
    Mi sono occupato di configurare opportunamente l'ambiente di CI scelto in modo da poter verificare la correttezza di ogni singola build, compilando ad ogni push su ogni branch e pull request.
    Inoltre, documentandomi online, ho trovato molto utile anche il sito \textit{Codecov.io}, che si occupa di mantenere delle statistiche sulla copertura dei test sul progetto.
    Per il rilascio della Doc e Scaladoc ad ogni rilascio sul branch di dev ho scelto \textit{Github Pages} in modo da renderlo disponibile per tutti, mentre per il rilascio dei pacchetti eseguibili degli applicativi server e client ad ogni push etichettata sul master ho scelto \textit{Github Releases}.
    Questa parte ha richiesto molto lavoro ancora prima di iniziare a sviluppare, ma successivamente tutto il team ne ha tratto beneficio.
    In corso d'opera, si e trattato solamente di adattare mano a mano la configurazione già esistente alle crescenti esigenze del progetto (sopratutto in quanto a riduzione dei tempi di compilazione).

    \subsubsection{Build automation}
    Mi sono occupato inoltre degli script necessari per eseguire agilmente una compilazione di tutto il progetto o di parte di esso in base alle esigenze.
    Tutto il progetto è stato incapsulato in moduli logicamente separati, seppur talvolta con dipendenze gli uni dagli altri.
    Questo ha permesso tante volte di ridurre tempi morti, compilando o testando solamente parti del progetto.

    \subsubsection{Core}
    Assieme a Lorenzo mi sono occupato dello sviluppo delle entità base del progetto e di tutti quegli elementi di gioco che riguardavano le regole del gioco e l'interazione tra di esse.

    \newpage


    \section{Retrospettiva}\label{sec:retrospettiva}
    \subsection{Organizzazione del processo di sviluppo}
Il processo di sviluppo è stato suddiviso in un primo sprint iniziale per confrontarsi sul design architetturale da adottare e discutere delle interazioni che i vari componenti dovevano avere.
In questo sprint abbiamo poi definito i quattro successivi di sviluppo del progetto.
Ogni sprint (eccetto quello iniziale) ha avuto mediamente una durata di circa 1,5/2 settimane.

\subsubsection{Preparazione iniziale}
Durante questo fase ci siamo incontrati più volte per definire l’architettura e i requisiti base che il nostro progetto doveva soddisfare.
Inoltre abbiamo individuato i componenti di gioco e le loro interazioni.
In questo modo abbiamo avuto l'opportunità di confrontarci e discutere per una migliore organizzazione dei successivi sprint.
Abbiamo inoltre configurato l’ambiente di sviluppo e assegnatoci i rispettivi compiti in base all’interesse di ognuno.
Tutta questa fase è stata completamente da remoto con lunghe chiamate su Microsoft Teams, tranne per l’ultima riunione in cui siamo riusciti a vederci tutti insieme.

\subsubsection{Sprint 1}
Questo sprint consisteva nel sviluppare le seguenti funzionalità:
\begin{itemize}
    \item Visualizzazione di una view di base della lobby in cui l’utente può inserire un username e il numero di giocatori con cui vuole partecipare.
    \item Per la parte server, creazione della lobby e gestione degli utenti in coda fino al raggiungimento necessario dei numeri di giocatori.
    \item Per il core, modellazione delle entità di gioco sia in Scala che in Prolog: colore, seme, valore, carte e le regole di validazione.
\end{itemize}
In questa fase abbiamo notato che il carico di lavoro per il facimento del Core e del Prolog è stato maggiore rispetto agli altri, dato che su di essi si sarebbe basato molto del lavoro del successivo sprint, causando una maggiore pressione nei due membri del team a cui è stato assegnato questo compito.

\subsubsection{Sprint 2}
Questo sprint consisteva nel sviluppare le seguenti funzionalità:
\begin{itemize}
    \item Creazione della struttura del client, collegamento del client con server durante la fase di registrazione ad una lobby.
    \item Inizio gioco: collegamento della lobby con lo stato iniziale della partita.
    \item View di gioco con la visualizzazione del tavolo e delle carte in mano ad un giocatore.
    \item Definizione dell’interfaccia core per la gestione del gioco.
    In questo sprint abbiamo deciso di arricchire la libreria prolog con l’aggiunta dell’ordinamento delle carte per seme e per valore che fino a quel momento erano state implementate in scala.
\end{itemize}

\subsubsection{Sprint 3}
L’obiettivo di questo sprint era quello di collegare le singole parti sviluppate in modo da ottenere un collegamento tra client-core-server e view visualizzando le funzionalità delle partita.
Abbiamo terminato tutta l’implementazione del Core con qualche piccolo aggiustamento.
Da questo sprint in poi sono iniziati ad essere rilevati tanti problemi con la validazione delle combinazioni in Prolog, causati dalla molteplice valenza del valore Asso (come primo e ultimo valore di una scala e possibilità di essere presente in due copie).
\newline
Questo è stato lo sprint più importante di tutto il processo di sviluppo poiché sono stati completati molti task alla gestione del gioco.
Inoltre, abbiamo notato il problema inverso rispetto al primo sprint: il carico di lavoro sul Server e sul Client è stato maggiore che sul resto.
Tuttavia in questo caso è stato possibile per i due membri restanti aiutare gli altri nella realizzazione dei loro task.

\subsubsection{Sprint 4}
Nell’ultimo sprint ci siamo occupati di risolvere alcuni problemi legati al server come la disconnessione improvvisa da parte di un utente o l’abbandono della partita.
Inoltre, lato client, abbiamo implementato la gestione del tempo che ogni giocatore ha a disposizione per completare il turno.
In questa fase tutti i membri del team hanno contribuito a risolvere i bug e rifattorizzare il codice presente.

\subsection{Sviluppi futuri}
Di sicuro come prossimi sviluppi potrebbero essere affrontati tutti i punti opzionali che non siamo riusciti a svolgere entro il tempo del progetto, quali:
\begin{itemize}
    \item Funzionalità che suggerisca all'utente una possibile mossa da fare nel proprio turno.
    \item Semplice AI contro cui giocare in locale.
    \item Possibilità di personalizzare alcune caratteristiche di questo gioco.
\end{itemize}
Al di fuori di essi, principalmente si è pensato a:
\begin{itemize}
    \item modificare il server per poterlo installare su di un container da mettere su una macchina remoto raggiungibile da remoto in modo da poter estendere i possibili utenti utilizzatori del gioco, per creare partite più avvincenti ed un’esperienza d’uso più completa.
    \item creare una base dati per la memorizzazione dei risultati delle partite e degli utenti in modo da generare periodicamente una classifica dei migliori giocatori.
    \item migliorare la funzionalità di generazione di partite private per arrivare a creare dei veri e propri tornei.
    Gli utenti, inseriti dentro a delle maxi lobby private, all’avvio del turno di un torneo, verrebbero accoppiati ad altri giocatori in base ai risultati delle partite precedenti.
    Gli utenti tornerebbero nella stessa maxi lobby al termine del turno per poter di nuovo essere accoppiati.
    Regole di gestione del torneo ulteriori, come ad esempio i gironi all’italiana o i punti della classifica potrebbero essere scelte direttamente dai gestori del torneo.
\end{itemize}

\subsection{Commenti finali}

\end{document}
