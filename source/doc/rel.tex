\documentclass{article}

\usepackage[utf8]{inputenc}
\usepackage{graphicx}

% TODO: Resolve unsupported character.
\title{Scalavelli project}
\date{25 agosto 2020}
\author{Daniele Tentoni "Part Time Team"}

\begin{document}
    \pagenumbering{gobble} % no numbers.
    \maketitle
    \newpage
    \pagenumbering{arabic} % start counting numbers.

    \begin{abstract}
        Progetto scritto in Scala per giocare al celebre gioco Machivelli sul proprio computer e sfidare altri giocatori.
    \end{abstract}

    \tableofcontents

    \newpage


    \section{Processo di sviluppo}

    \subsection{Metodologia}

    Sin da subito abbiamo deciso di adottare una metodologia di sviluppo \textit{Agile-Scrum}, seppur non scegliendo un
    vero e proprio Scrum Master. Chi più e chi meno, a seconda dei vari sprint, praticamente tutti hanno avuto
    l'occasione, il modo e la motivazione di ricoprire tale ruolo, in modo tale da favorire il proseguimento del
    progetto coordinando il team con l'aiuto degli strumenti a disposizione. % Chi è il nostro Scrum Master?
    In accordo con tale metodologia, il lavoro è stato suddiviso in sprint al termine dei quali si sarebbero dovute
    sviluppare un numero minimo di funzionalità dell'applicativo, della durata media di una settimana e mezzo. I meeting
    sono stati frequenti all'inizio e ne sono stati svolti alcuni saltuariamente all'interno di ogni sprint. Questo
    perché, alternandosi con un periodo di lavoro autonomo, abbiamo ritenuto necessario e producente confrontarsi anche
    durante gli sprint su scelte sintattiche e pattern di sviluppo tra tutti i membri del team, in modo da condividere
    conoscenze ed entusiasmo.

    \subsection{Strumenti adottati}
    Come strategia di workflowing è stata scelta questa variante molto semplice, schematizzata di seguito:
    % TODO: Produrre immagine che rappresenti il workflow.
    Come strumenti di \textit{Build Automation} si è deciso di usare \textit{SBT}, dato che durante il corso lo si è
    sempre preferito rispetto al cugino Gradle per lo svolgimento degli elaborati, nonostante il fatto che alcuni
    componenti del gruppo lo uilizzino in ambito aziendale.
    La comunicazione all'interno del gruppo è avvenuta su \textit{Telegram} per quanto concerne meri aspetti
    organizzativi, mentre la condivisione di appunti e blocchi di codice durante le sessioni di sviluppo è stato
    sfruttato \textit{Microsoft Teams}.
    Per la \textit{Continuous Integration} è stato scelto \textit{Travis CI}. Consisteva nell'unica soluzione freeware
    che i componenti del gruppo abbiano mai usato, introdotta proprio in questo corso. Tuttavia, è stato necessario
    approfondire l'argomento tramite studio autonomo, dato che ciò che si era appreso a lezione è stato ritenuto
    insufficente per la buona riuscita del progetto per come è stato pensato. Ad esso è stata affidata l'esecuzione dei
    test necessari per verificare la correttezza del lavoro svolto e poter poi effettuare dei rilasci senza regressioni.
    Tramite lo stesso servizio, è stata effettuata la \textit{Continuous Delivery}, rilasciando dei pacchetti compilati
    su \textit{Github Releases}.

    \newpage


    \section{Requisiti}

    \subsection{Business}

    \subsection{Utente}

    \subsection{Funzionali}

    \subsection{Non Funzionali}

    \subsection{Implementativi}

    \newpage


    \section{Design Architetturale}

    \subsection{Model}

    \subsection{View}

    \subsection{Controller}

    \newpage


    \section{Design di Dettaglio}

    \newpage


    \section{Implementazione}

    \subsection{Matteo}

    \subsection{Luca}

    \subsection{Lorenzo}

    \subsection{Daniele}

    \subsubsection{Continuous Integration}
    Io mi sono occupato di configurare opportunamente l'ambiente di CI scelto in modo da poter verificare la
    correttezza di ogni singola build, compilando ad ogni push su ogni branch e pull request. Inoltre, viene
    effettuato anche un upload sul sito Codecov.io (che si occupa di mantenere delle statistiche sulla copertura
    dei test sul progetto), del rilascio della Doc e Scaladoc ad ogni rilascio sul branch di sviluppo sull'ambiente
    di Github-Pages in modo da renderlo disponibile per tutti e di effettuare un rilascio dei pacchetti eseguibili
    degli applicativi server e client ad ogni push sul master che sia stata etichettata da un tag. Questa parte
    ha richiesto molto lavoro ancora prima di iniziare a sviluppare, ma successivamente tutto il team ne ha tratto
    beneficio. Successivamente, in corso d'opera, si è trattato solamente di adattare mano a mano la configurazione
    già esistente alle crescenti esigenze del progetto. In primo luogo, il bisogno di stringere i tempi di compilazione
    sull'ambiente remoto, che già all'inizio del progetto, quando quindi era ancora relativamente piccolo, iniziavano
    già a richiedere diversi minuti per ogni singolo passaggio, andandosi a sommare in lunghe compilazioni tra i
    10 e i 15 minuti. Successivamente ha richiesto di risolvere qualche problema con la compilazione e l'impacchettamento
    degli eseguibili.

    \subsubsection{Build automation}
    Mi sono occupato inoltre degli script necessari per eseguire agilmente una compilazione di tutto il progetto
    o di parte di esso in base alle esigenze. Ho configurato tutto il progetto in modo che fosse logicamente diviso
    in moduli in modo da aumentare l'incapsulamente e l'esposizione ad altre porzioni di progetto di solo le parti
    necessarie. Ho inoltre configurato tutti i pacchetti e le librerie aggiuntive da importare nel progetto.

    \subsubsection{Model}
    Assieme a Lorenzo mi sono occupato dello sviluppo delle entità base del progetto e di tutti quegli elementi
    di gioco che riguardavano le regole del gioco e l'iterazione tra di esse.

    \newpage


    \section{Retrospettiva}

    \subsection{Problemi riscontrati}

    \subsection{Sviluppi futuri}

    \newpage

\end{document}
