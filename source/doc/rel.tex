\documentclass{article}

\title{Scalavelli project}
\date{25 agosto 2020}
\author{Daniele Tentoni <Part Time Team>}

\begin{document}
    \pagenumbering{gobble} % no numbers.
    \maketitle
	\newpage
	\pagenumbering{arabic} % start counting numbers.

    \begin{abstract}
        Progetto scritto in Scala per giocare al celebre gioco Machivelli sul proprio computer e sfidare altri giocatori.
    \end{abstract}

    \tableofcontents
    
    \section{Processo di sviluppo}

	\newpage

    \section{Requisiti}

        \subsection{Business}

        \subsection{Utente}

        \subsection{Funzionali}

        \subsection{Non Funzionali}

        \subsection{Implementativi}

	\newpage
    
    \section{Design Architetturale}

        \subsection{Model}

        \subsection{View}

        \subsection{Controller}

	\newpage

    \section{Design di Dettaglio}

	\newpage

    \section{Implementazione}

        \subsection{Matteo}

        \subsection{Luca}

        \subsection{Lorenzo}

        \subsection{Daniele}

            \subsubsection{Continuous Integration}
            Io mi sono occupato di configurare opportunamente l'ambiente di CI scelto in modo da poter verificare la
            correttezza di ogni singola build, compilando ad ogni push su ogni branch e pull request. Inoltre, viene
            effettuato anche un upload sul sito Codecov.io (che si occupa di mantenere delle statistiche sulla copertura
            dei test sul progetto), del rilascio della Doc e Scaladoc ad ogni rilascio sul branch di sviluppo sull'ambiente
            di Github-Pages in modo da renderlo disponibile per tutti e di effettuare un rilascio dei pacchetti eseguibili
            degli applicativi server e client ad ogni push sul master che sia stata etichettata da un tag. Questa parte
            ha richiesto molto lavoro ancora prima di iniziare a sviluppare, ma successivamente tutto il team ne ha tratto
            beneficio. Successivamente, in corso d'opera, si è trattato solamente di adattare mano a mano la configurazione
            già esistente alle crescenti esigenze del progetto. In primo luogo, il bisogno di stringere i tempi di compilazione
            sull'ambiente remoto, che già all'inizio del progetto, quando quindi era ancora relativamente piccolo, iniziavano
            già a richiedere diversi minuti per ogni singolo passaggio, andandosi a sommare in lunghe compilazioni tra i
            10 e i 15 minuti. Successivamente ha richiesto di risolvere qualche problema con la compilazione e l'impacchettamento
            degli eseguibili.

            \subsubsection{Build automation}
            Mi sono occupato inoltre degli script necessari per eseguire agilmente una compilazione di tutto il progetto
            o di parte di esso in base alle esigenze. Ho configurato tutto il progetto in modo che fosse logicamente diviso
            in moduli in modo da aumentare l'incapsulamente e l'esposizione ad altre porzioni di progetto di solo le parti
            necessarie. Ho inoltre configurato tutti i pacchetti e le librerie aggiuntive da importare nel progetto.

            \subsubsection{Model}
            Assieme a Lorenzo mi sono occupato dello sviluppo delle entità base del progetto e di tutti quegli elementi
            di gioco che riguardavano le regole del gioco e l'iterazione tra di esse.

	\newpage

    \section{Retrospettiva}

        \subsection{Problemi riscontrati}

        \subsection{Sviluppi futuri}

	\newpage

\end{document}
