\subsection{Organizzazione del processo di sviluppo}
Il processo di sviluppo è stato suddiviso in un primo sprint iniziale per confrontarsi sul design architetturale da adottare e discutere delle interazioni che i vari componenti dovevano avere.
In questo sprint abbiamo poi definito i quattro successivi di sviluppo del progetto.
Ogni sprint (eccetto quello iniziale) ha avuto mediamente una durata di circa 1,5/2 settimane.

\subsubsection{Preparazione iniziale}
Durante questo fase ci siamo incontrati più volte per definire l’architettura e i requisiti base che il nostro progetto doveva soddisfare.
Inoltre abbiamo individuato i componenti di gioco e le loro interazioni.
In questo modo abbiamo avuto l'opportunità di confrontarci e discutere per una migliore organizzazione dei successivi sprint.
Abbiamo inoltre configurato l’ambiente di sviluppo e assegnatoci i rispettivi compiti in base all’interesse di ognuno.
Tutta questa fase è stata completamente da remoto con lunghe chiamate su Microsoft Teams, tranne per l’ultima riunione in cui siamo riusciti a vederci tutti insieme.

\subsubsection{Sprint 1}
Questo sprint consisteva nel sviluppare le seguenti funzionalità:
\begin{itemize}
    \item Visualizzazione di una view di base della lobby in cui l’utente può inserire un username e il numero di giocatori con cui vuole partecipare.
    \item Per la parte server, creazione della lobby e gestione degli utenti in coda fino al raggiungimento necessario dei numeri di giocatori.
    \item Per il core, modellazione delle entità di gioco sia in Scala che in Prolog: colore, seme, valore, carte e le regole di validazione.
\end{itemize}
In questa fase abbiamo notato che il carico di lavoro per il facimento del Core e del Prolog è stato maggiore rispetto agli altri, dato che su di essi si sarebbe basato molto del lavoro del successivo sprint, causando una maggiore pressione nei due membri del team a cui è stato assegnato questo compito.

\subsubsection{Sprint 2}
Questo sprint consisteva nel sviluppare le seguenti funzionalità:
\begin{itemize}
    \item Creazione della struttura del client, collegamento del client con server durante la fase di registrazione ad una lobby.
    \item Inizio gioco: collegamento della lobby con lo stato iniziale della partita.
    \item View di gioco con la visualizzazione del tavolo e delle carte in mano ad un giocatore.
    \item Definizione dell’interfaccia core per la gestione del gioco.
    In questo sprint abbiamo deciso di arricchire la libreria prolog con l’aggiunta dell’ordinamento delle carte per seme e per valore che fino a quel momento erano state implementate in scala.
\end{itemize}

\subsubsection{Sprint 3}
L’obiettivo di questo sprint era quello di collegare le singole parti sviluppate in modo da ottenere un collegamento tra client-core-server e view visualizzando le funzionalità delle partita.
Abbiamo terminato tutta l’implementazione del Core con qualche piccolo aggiustamento.
Da questo sprint in poi sono iniziati ad essere rilevati tanti problemi con la validazione delle combinazioni in Prolog, causati dalla molteplice valenza del valore Asso (come primo e ultimo valore di una scala e possibilità di essere presente in due copie).
\newline
Questo è stato lo sprint più importante di tutto il processo di sviluppo poiché sono stati completati molti task alla gestione del gioco.
Inoltre, abbiamo notato il problema inverso rispetto al primo sprint: il carico di lavoro sul Server e sul Client è stato maggiore che sul resto.
Tuttavia in questo caso è stato possibile per i due membri restanti aiutare gli altri nella realizzazione dei loro task.

\subsubsection{Sprint 4}
Nell’ultimo sprint ci siamo occupati di risolvere alcuni problemi legati al server come la disconnessione improvvisa da parte di un utente o l’abbandono della partita.
Inoltre, lato client, abbiamo implementato la gestione del tempo che ogni giocatore ha a disposizione per completare il turno.
In questa fase tutti i membri del team hanno contribuito a risolvere i bug e rifattorizzare il codice presente.

\subsection{Sviluppi futuri}
Di sicuro come prossimi sviluppi potrebbero essere affrontati tutti i punti opzionali che non siamo riusciti a svolgere entro il tempo del progetto, quali:
\begin{itemize}
    \item Funzionalità che suggerisca all'utente una possibile mossa da fare nel proprio turno.
    \item Semplice AI contro cui giocare in locale.
    \item Possibilità di personalizzare alcune caratteristiche di questo gioco.
\end{itemize}
Al di fuori di essi, principalmente si è pensato a:
\begin{itemize}
    \item modificare il server per poterlo installare su di un container da mettere su una macchina remoto raggiungibile da remoto in modo da poter estendere i possibili utenti utilizzatori del gioco, per creare partite più avvincenti ed un’esperienza d’uso più completa.
    \item creare una base dati per la memorizzazione dei risultati delle partite e degli utenti in modo da generare periodicamente una classifica dei migliori giocatori.
    \item migliorare la funzionalità di generazione di partite private per arrivare a creare dei veri e propri tornei.
    Gli utenti, inseriti dentro a delle maxi lobby private, all’avvio del turno di un torneo, verrebbero accoppiati ad altri giocatori in base ai risultati delle partite precedenti.
    Gli utenti tornerebbero nella stessa maxi lobby al termine del turno per poter di nuovo essere accoppiati.
    Regole di gestione del torneo ulteriori, come ad esempio i gironi all’italiana o i punti della classifica potrebbero essere scelte direttamente dai gestori del torneo.
\end{itemize}

\subsection{Commenti finali}